\documentclass[a4paper,9pt]{article}

\usepackage[margin=0.95in]{geometry}
\usepackage{inconsolata}
\usepackage[normalem]{ulem}
% \usepackage{fontspec}
\usepackage{charter}
% \setromanfont{Times New Roman}
\usepackage[hidelinks,backref]{hyperref} % clickable links and citations with no green borders
\usepackage{amsmath}
\usepackage{listings} % add source code snippets
\usepackage{csquotes} % block quotes
\usepackage{color}

\usepackage{tabularx}

% use CJK chars
% \usepackage{xeCJK}
% \setCJKmainfont{MS Gothic}

\usepackage[dvips]{graphicx}
\DeclareGraphicsExtensions{.png,.jpg}
\setlength{\parindent}{0pt}
\definecolor{mygreen}{rgb}{0,0.6,0}
\definecolor{mygray}{rgb}{0.5,0.5,0.5}
\definecolor{mydarkgray}{rgb}{0.4,0.4,0.4}
\definecolor{mymauve}{rgb}{0.58,0,0.82}
\definecolor{myrust}{rgb}{0.77,0,0}
\definecolor{darkblue}{rgb}{0.0,0.0,0.3}
\pagestyle{empty}

\lstset{ %
  backgroundcolor=\color{white},   % choose the background color; you must add \usepackage{color} or \usepackage{xcolor}
  basicstyle=\footnotesize\ttfamily,        % the size of the fonts that are used for the code
  breakatwhitespace=false,         % sets if automatic breaks should only happen at whitespace
  breaklines=true,                 % sets automatic line breaking
  captionpos=b,                    % sets the caption-position to bottom
  commentstyle=\color{mygreen},    % comment style
  deletekeywords={...},            % if you want to delete keywords from the given language
  escapeinside={\%*}{*)},          % if you want to add LaTeX within your code
  extendedchars=true,              % lets you use non-ASCII characters; for 8-bits encodings only, does not work with UTF-8
  frame=single,                    % adds a frame around the code
  keepspaces=true,                 % keeps spaces in text, useful for keeping indentation of code (possibly needs columns=flexible)
  keywordstyle=\color{blue},       % keyword style
  language=C++,                    % the language of the code
  morekeywords={*,...},            % if you want to add more keywords to the set
  numbers=none,                    % where to put the line-numbers; possible values are (none, left, right)
  numbersep=5pt,                   % how far the line-numbers are from the code
  numberstyle=\tiny\color{mygray}, % the style that is used for the line-numbers
  rulecolor=\color{black},         % if not set, the frame-color may be changed on line-breaks within not-black text (e.g. comments (green here))
  showspaces=false,                % show spaces everywhere adding particular underscores; it overrides 'showstringspaces'
  showstringspaces=false,          % underline spaces within strings only
  showtabs=false,                  % show tabs within strings adding particular underscores
  stepnumber=2,                    % the step between two line-numbers. If it's 1, each line will be numbered
  stringstyle=\color{mymauve},     % string literal style
  tabsize=2,                       % sets default tabsize to 2 spaces
  % title=\lstname                   % show the filename of files included with \lstinputlisting; also try caption instead of title
}

\hypersetup{
  colorlinks=true,
  linkcolor=red,
  urlcolor=darkblue,
  citecolor=green,
  linktoc=page
}

\begin{document}

% \vspace*{4.6cm}

\begin{tabularx}{\textwidth}{p{0.48\textwidth}>{\raggedleft\arraybackslash}p{0.48\textwidth}}
\textcolor{myrust}{\Huge{Ranveer Aggarwal}} & \textcolor{mydarkgray}{Room 148, Hostel 3, IIT Bombay}\\
 & \textcolor{mydarkgray}{Powai 400076, Mumbai, India}\\
 & \href{mailto:ranveer@cse.iitb.ac.in}{\textcolor{blue}{ranveer@cse.iitb.ac.in}}\\
 & \href{http://www.ranveeraggarwal.com}{\textcolor{blue}{www.ranveeraggarwal.com}}\\
\end{tabularx}

\vspace{31pt}

% font sizes: http://www.emerson.emory.edu/services/latex/latex_169.html#SEC169

\textcolor{myrust}{\large{\textsc{About}}}\textcolor{mygray}{\sout{\hfill}}\\\\
\small %\normalsize
$\bullet$ Pursuing \textbf{B. Tech with Honors} in Computer Science and Engineering at IIT Bombay\\
$\bullet$ \textbf{Interests:} Computer Graphics, Computer and Network Security, Web Development\\
$\bullet$ \textbf{GPA:} 7.25/10.00 (after 5 semesters)\\\\

\textcolor{myrust}{\large{\textsc{Work Experience}}}\textcolor{mygray}{\sout{\hfill}}\\\\
\small %\normalsize
$\bullet$ \textbf{Season of KDE Intern}, PlanetKDE\hfill \textit{Winter 2014}\\
\textit{Mentor}: \href{http://jriddell.org/about-me/}{\textcolor{mydarkgray}{Jonathan Riddell}}\\
\hspace*{0.35cm}$\circ$ Redesigned and redeveloped \href{https://planetkde.org/}{PlanetKDE}, KDE’s blog aggregator built on rawdog\\
\hspace*{0.35cm}$\circ$ Built a mobile friendly, KDE design scheme compliant, flat interface with social media plugins, working closely with KDE’s design team and the dev community\\\\

\textcolor{myrust}{\large{\textsc{Key Academic Projects}}}\textcolor{mygray}{\sout{\hfill}}\\\\
\small %\normalsize
\textbf{$\bullet$} \href{https://github.com/ranveeraggarwal/lightcuts}{\textbf{Lightcuts}}\hfill\textit{Spring 2015}\\
\hspace*{0.35cm}$\circ$ Implemented the research paper \href{http://www.cs.cornell.edu/~kb/projects/lightcuts/}{`Lightcuts' [SIGGRAPH 2005]} in a team of two for the Advanced Computer Graphics course project. The paper describes a scalable framework for computing realistic illumination\\
\hspace*{0.35cm}$\circ$ Developed it as a plugin for \href{http://www.pbrt.org/}{\textsl{pbrt-v2}}, an open source renderer based on the book `Physically Based Rendering'\\
\textbf{$\bullet$} \href{https://github.com/ranveeraggarwal/renderman-transformer/}{\textbf{Rendering with Photorealistic Renderman}} \hfill \textit{Spring 2015}\\
\hspace*{0.35cm}$\circ$ Wrote shaders and rendered raytraced scenes using Pixar's rendering software PRMan\\
\hspace*{0.35cm}$\circ$ The resultant scene elements produced effects like color bleeding, caustics, area lights and soft shadows\\
\textbf{$\bullet$} \href{https://github.com/ranveeraggarwal/transformer}{\textbf{Transformer Rendering and Animation}} \hfill \textit{Autumn 2014}\\
\hspace*{0.35cm}$\circ$ Modeled, textured and animated (forward kinematics) a transformer robot from scratch with OpenGL\\
\hspace*{0.35cm}$\circ$ Developed an interactive environment for the keyboard controlled bot with inter-object collisions\\
\hspace*{0.35cm}$\circ$ Used motion captured data in the form of BVH inputs to animate the transformer (reverse kinematics)\\
\textbf{$\bullet$} \href{https://github.com/ranveeraggarwal/chess-titans}{\textbf{Chess with Artificial Intelligence}}\hfill \textit{Spring 2013}\\
\hspace*{0.35cm}$\circ$ Developed a chess game in PLT Scheme using in-built GUI Toolkit in DrRacket\\
\hspace*{0.35cm}$\circ$ Implemented the Minimax Algorithm with Alpha-Beta Pruning for the AI with a tree depth of 3\\
\textbf{$\bullet$} \href{https://github.com/ranveeraggarwal/compiler}{\textbf{Incremental Development of a Compiler}}\hfill \textit{Spring 2015}\\
\hspace*{0.35cm}$\circ$ Designed a compiler incrementally with different stages for tokenizing, parsing, AST-generation, semantic analysis and finally, machine code generation using FlexC++ and BisonC++\\
\textbf{$\bullet$} \textbf{E-Learning Academy (MOOC Platform)}\hfill \textit{Summer 2014}\\
\hspace*{0.35cm}$\circ$ Developed plugins and fixed bugs for the existing web-platform built in Django\\
\hspace*{0.35cm}$\circ$ Analysed user behaviour through data-logging and optimised the existing codebase\\
\hspace*{0.35cm}$\circ$ Based on the flipped-classroom model, the platform promotes student-centred learning, collaboration and improves content accessibility\\
\textbf{$\bullet$} \href{https://github.com/ranveeraggarwal/orrery-simulation-box2d}{\textbf{2D Simulation of an Orrery}}\hfill \textit{Spring 2014}\\
\hspace*{0.35cm}$\circ$ Simulated a mechanical model of Solar System using gears instead of gravity\\
\hspace*{0.35cm}$\circ$ Used Box2D, an open source physics engine for interaction between mechanical components\\\\

\textcolor{myrust}{\large{\textsc{Side Projects}}}\textcolor{mygray}{\sout{\hfill}}\\\\
\textbf{$\bullet$} \href{https://github.com/manishearth/kapi}{\textbf{Kapi - A Classroom Note Taker}}\hfill \textit{Spring 2014}\\
\hspace*{0.35cm}$\circ$ Designed an app that, along with normal text, typesets maths in LATEX format\\
\hspace*{0.35cm}$\circ$ Worked in a team of 4 to develop a program that recursively breaks down the \LaTeX{} chunks into smaller components and parses them at the token level\\
\hspace*{0.35cm}$\circ$ The application won the $1^{st}$ place at Microsoft Code.Fun.Do, 2014 and is currently live on the Windows App Store\\\\

\pagebreak

\textcolor{myrust}{\large{\textsc{Seminars}}}\textcolor{mygray}{\sout{\hfill}}\\\\
\small %\normalsize
$\bullet$ \href{http://www.cs.cornell.edu/~kb/publications/SIG12BidirLC.pdf}{\textbf{Bidirectional Lightcuts}}\hfill\textit{Spring 2015}\\
\textcolor{mydarkgray}{\textit{Advanced Computer Graphics course}}\hfill\textit{Guide}: \textcolor{mydarkgray}{Prof Parag Chaudhuri}\\
If real-world scenes are to be modelled, we need a fast, noise free rendering algorithm that handles all kinds of materials like glossy materials, and phenomenon like subsurface scattering. General unbiased algorithms like Path Tracing produce a lot of noise whereas specialised noise free algorithms like Instant Radiosity are biased, meaning several important illumination features might be missing. The paper, an extension of a previous paper titled `Lightcuts', implemented by the same author extends support to a wider variety of materials and phenomenon, while maintaining scalability and low noise. It uses clever weighing strategies to lower the bias in VPL-based algorithms and demonstrates scalable, efficient, and low noise rendering of scenes with highly complex materials including gloss, BSSRDFs, and anisotropic volumetric models. This was presented in a team of 2 for an advanced computer graphics course.\\\\
$\bullet$ \href{http://rbr.cs.umass.edu/papers/HZaij01b.pdf}{\textbf{LAO*: A Heuristic Search Algorithm That Finds Solution with Loops}}\hfill\textit{Spring 2015}\\
\textcolor{mydarkgray}{\textit{Artificial Intelligence course}}\hfill\textit{Guide}: \textcolor{mydarkgray}{Prof Pushpak Bhattacharya}\\
Classic heuristic search algorithms can find solutions that take the form of a simple path (A*), a tree, or an acyclic graph (AO*). This paper describes a novel generalization of heuristic search, called LAO*, that can find solutions with loops. It is shown that LAO* can be used to solve Markov decision problems and that it shares the advantage heuristic search has over dynamic programming for other classes of problems. Given a start state, it finds an optimal solution without evaluating the entire state space. This paper was presented as a part of an Artificial Intelligence course.\\\\

\textcolor{myrust}{\large{\textsc{Achievements}}}\textcolor{mygray}{\sout{\hfill}}\\\\
\small %\normalsize
\textbf{National}\\
$\bullet$ Part of the team (of four) that stood third for two consecutive years at the national inter-collegiate hacking competitions, \textsl{Build the Shield 2015 and HackCon 2014} organised by Microsoft\\ 
$\bullet$ Bagged the first position in a team of four at both institute and national level at Code.Fun.Do 2014, a hackathon cum accelerator program by Microsoft India Development Center\\ 
$\bullet$ Attained an All India Rank of 104 (State Rank 2) among 3.75 lakh participants in National Level Science Talent Search Examination (NSTSE) 2012\\ 
$\bullet$ Secured All India Rank 1 in International Olympiad of Science (IOS) 2009\\ 
$\bullet$ Achieved an All India Rank of 53 in National Science Olympiad (NSO) 2008\\\\
\textbf{Institute Level}\\
$\bullet$ Stood first in autonomous line follower robotics competition for freshmen amongst 50+ teams\\ 
$\bullet$ Secured third position in RC Car building competition among over 100 teams\\ 
$\bullet$ Stood second in institute-level remote-controlled football-playing bot making competition\\\\

\textcolor{myrust}{\large{\textsc{Technical Skills}}}\textcolor{mygray}{\sout{\hfill}}\\\\
\small %\normalsize
$\bullet$ \textbf{Knowledgeable about} C/C\verb!++!, OpenGL, Python, HTML, CSS, DrRacket\\
$\bullet$ \textbf{Basic familiarity with} Java, JavaScript, PHP, Renderman, VHDL, Bash, MIPS-Assembly\\\\

\textcolor{myrust}{\large{\textsc{Campus Activities}}}\textcolor{mygray}{\sout{\hfill}}\\\\
\small %\normalsize
$\bullet$ \textbf{Web and Coding Club, IIT Bombay}\hfill\textit{May 2013 -- April 2015}\\
\hspace*{0.35cm}$\circ$ As a manager, led a two-tier team consisting of 9 co-ordinators to encourage programming as a hobby\\
\hspace*{0.35cm}$\circ$ Mentored 15 freshmen teams under Institute Technical Summer Projects out of which 9 successfully completed their projects and 3 came up with prototypes\\
$\bullet$ \textbf{National Sports Organisation} \hfill\textit{July 2012 -- April 2013}\\
\hspace*{0.35cm}$\circ$ Completed the year-long course by National Sports Organization in Squash\\\\

% \Large{\textsc{Additional Courses Taken}}\vspace{1.5pt}
% \large{\textsc{Additional Courses Taken}}\vspace{1.5pt}
% \hrule\vspace{0.25cm}
% \textcolor{myrust}{% \large{\textsc{Programming and Technical Skills}}}\textcolor{mygray}{\sout{\hfill}}\\\\
% \small %\normalsize
% \textbf{CS:} Digital Image Processing\textsuperscript{1}, Parallelizing Compilers\textsuperscript{1}, Parallel Computation\textsuperscript{1}, Advanced Computer Graphics, Computer Graphics, Machine Learning, Program Derivation\\
% \textbf{Others:} Linear and Nonlinear Systems, Signals and Feedback Systems, Mathematical Structures for Systems and Controls\\
% % \textsuperscript{1}\emph{With associated lab course}
% \null\hfill\textsuperscript{1}\emph{Ongoing, will be completed by Dec 2014}\\
% % see http://tex.stackexchange.com/a/75145 for an explanation of why \null\hfill works while \hfill doesn't

\textcolor{myrust}{\large{\textsc{Hobbies and Interests}}}\textcolor{mygray}{\sout{\hfill}}\\\\
\small %\normalsize
$\bullet$ Contributor to several KDE projects including Krita and PlanetKDE\\
$\bullet$ Developed an application, titled \href{https://github.com/ranveeraggarwal/Rumor-Roll}{Rumor Roll!} in php using Yahoo! Boss API and YQL that outputs rumours related to the given query at Yahoo! HackU 2013.\\
$\bullet$ Built a JavaScript based game \href{https://github.com/ranveeraggarwal/fission}{Fission}, on the lines of popular game, Chain Reaction\\
$\bullet$ Enthusiastic in swimming and water adventure sports\\
\end{document}